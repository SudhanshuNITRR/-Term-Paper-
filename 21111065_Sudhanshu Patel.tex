\documentclass[12pt]{article}
\usepackage{graphicx}
\graphicspath{{images}}
\usepackage{adjustbox}
% Comment the following line to NOT allow the usage of umlauts
\usepackage[utf8]{inputenc}
% Uncomment the following line to allow the usage of graphics (.png, .jpg)
%\usepackage{graphicx}

% Start the document
\begin{document}
\tableofcontents
\clearpage
% Create a new 1st level heading
\section*{ABSTRACT}
This paper presents a simple and shorter theory of " PORTABLE X- RAYS MACHINE IN HEALTHCARE " and some important topics related to X-Ray Machine. A portable X-ray machine is one that can be moved and is
smaller than a fixed one. It enables radiographers, veterinarians, and dental
specialists to capture X-ray images of patients without requiring them to en-
ter a lead-lined room. The most frequent form of electric magnetic radiation
is X-rays. X-ray has been widely employed since its
discovery in 1895. The X-ray tube is a device that emits radiation. Each X-ray
union requires a vital component, which scientists and clinicians used in the
early phases. Gas ion tubing is a type of tube that is used to transport gas
ions When fast-moving electrons suddenly decelerate, X-rays are produced.
They collide with and interact with the target anode. As a result, in order
to obtain acceptable radiation output for digital radiography, an x-ray tube
absorbs and dissipates a large heat charge, altering the structure and func-
tion of an x-ray source. The essential components of an x-ray tube Frames
for electrodes and counter electrodes, rotor and stator, and tube cover are
all included. X-rays were taken. The waveforms created are comparable to
those of other electromagnetic waves. As x-rays become more common, The
tiny wavelength of radiation characterises their property over matter. The
most of the materials Radiation will not pass through lead or other thick
materials since they are transparent to x-rays.

\clearpage


\section{INTRODUCTION}
Wilhelm Conrad Röntgen, a German physicist, was the first to discover the technology behind X-ray machines. In 1895, Röntgen started experimenting cathode rays in a laboratory in Wurzberg, Germany (part of the technology behind what would later become television). According to the History Channel, he intended to investigate if the rays could pass through a glass "when he spotted a glow coming from a nearby chemically coated screen." "Because of their unknown nature, he called the rays that generated this illumination X-rays."
\\
During World War I, the scientist created a "Little Curie," a mobile x-ray equipment, and trained 150 women to operate it. X-ray machines that can be taken with you are called portable x-ray machines. Increases the efficiency of the workflow. Increases the number of patients who can be seen in a given time period. Visits with injured or terminally sick patients are made easier. Allows doctors to see patients off-site when they are in conditions that make it difficult or impossible for them to travel to a facility.
\section{WORKING OF X-RAY MACHINES }
The X-ray tube is activated by supplying a regulated voltage and current. As a result, altering voltage or current can control the X-ray beam intensity. The beam is focused on the target. Some beams will travel through the item, while others will be absorbed, from which we get X-Ray.
\\
The body is exposed to a little amount of ionising radiation. This used to be printed on a specific sheet of film. Nowadays, x-ray exams are more likely to involve the use of a gadget that captures transmitted x-rays and converts them to an electronic image Because calcium in bones prevents radiation from passing through, healthy bones appear white or grey. Radiation, on the other hand, easily flows through air voids, making healthy lungs appear black. Thus we get X-Ray.
\begin{figure}
\centering
\includegraphics[scale=0.5]{IMG-20220406-WA0007.jpg}
\caption{}
\end{figure}
\\
\section{BENIFITS OF PORTABLE X-RAYS MACHINE IN HEALTHCARE}
After the invention of the portable X-ray machine, there has been a change in the entire health department.  Patients no longer have to wait longer to get X-rays.  X-rays of more and more patients can be obtained simultaneously.
\begin{itemize}
\item MOBILITY AND PORTABILITY:
\\
The ability to stop transfers and further patient motions is a big benefit of having a small X-ray. Hospitalizations or trips to the radiology unit for chest, muscle, and belly X-ray evaluations can sometimes cause more harm than help to patients, especially those with serious medical issues.
During X-ray tests, these machines are used to remove undesired transportation and changes in the patient's body posture.
Most portable X-ray machines come with a wheeled platform, while some are motorised, allowing technicians to operate the unit without needing to plug it in.
\\
\item Speed :
\\
The response time of radiograph tests performed with handheld imaging tools is slower, but there are no extended wait times.
In just 20 minutes, you can use portable X-ray equipment. Furthermore, the reports can be triggered in real-time and transmitted directly to the patient's physician, which takes about an hour to complete, allowing for quick identification and treatment of the patient.
\begin{figure}
\centering
\includegraphics[scale=0.5]{IMG-20220406-WA0081.jpg}
\caption{}
\end{figure}
\item Security :
\\
When using traditional medical imaging technologies, radiation exposure remains one of the main health dangers for both patients and employees; nevertheless, data show that employing mobile radiology devices reduces radiation exposure dramatically.
The mask on the front of the portable X-ray equipment shields them from the radiation it emits.
\\
\item Economical Effectiveness :
\\
It is costly to transport patients. By reducing the use of ambulance and cab transportation, as well as staff pull-out to follow patients to and from the clinic, portable X-ray equipment can have socioeconomic benefits. Mobile X-rays help to reduce total costs.
\begin{figure}
\centering
\includegraphics[scale=0.5]{IMG-20220406-WA0082.jpg}
\caption{}
\end{figure}
\item Image Quality :
\\
Cutting-edge technology is used in portable optical x-ray equipment. This allows technologists to produce high-quality images in a short amount of time.
The optical scanner on the scanning device does not require film and instead uses phosphor trays and cassettes that may be readily erased while in operation.
Technologists can also improve digital clinical images captured in the region by connecting the workstation to the portable device.
\end{itemize}
\section{DISADVANTAGES OF PORTABLE X-RAYS MACHINE IN HEALTHCARE}
\begin{itemize}
\item 3D information is not provided by portable X-ray.
\item As the radiation is absorbed by the bones, it can cause considerable diagnostic data to be lost.
\item Lighter elements do not interact as strongly with X rays.
\item X-rays cause ionisation by mutating cells due to their radiation. This frequently results in cancer.
\item The best image produced by an x-ray is of medium quality.
\end{itemize}

\section{CONCLUSION}

The invention of x ray has brought change in the entire health department. Portable X-ray has proven to be an effective tool for diagnosing and monitoring patients in intensive care units, nursing homes, jails, and homeless shelters when access to a hospital radiology department is a challenge. Patients who are unable to be transported to the Radiology department are frequently treated with portable x-ray devices. As a result, x-ray equipment is built with certain qualities that allow it to be moved in tight places, such as between hospital beds in small wards.Portable X-Ray Machine is a very important and useful medical device. 

\section{REFERENCES}
\begin{itemize}
\item https://www.plus100years.com/read-more-blog/portable-x-ray-machines
\item youtube.com
\item https://en.wikipedia.org/wiki/X-ray-machine
\item https://images.app.goo.gl/veYEiyEZtjMtk6A6A
\end{itemize}


% Uncomment the following two lines if you want to have a bibliography
%\bibliographystyle{alpha}
%\bibliography{document}

\end{document}
